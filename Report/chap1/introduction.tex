

% --------------------------------------------------------------


\chapter{Introduction}
\section{Motivation}
It is most often that the observations have the power to push the well established field to its most stringent tests. One such observation was of the supernovae type-Ia (SNe-Ia) which established that the universe is expanding. After SNe-Ia observation, there has been number of other observations which establishes that the expansion is accelerating such as Baryon Acoustic Oscillations (BAO), Cosmic Microwave Background Radiation(CMBR), Gamma-ray Bursts (GRBs). The observational results of SNe-Ia together with CMBR power spectrum show that our universe is mainly made up of two components: dark matter and dark energy. The dark matter contribute one-0third of the total energy density of the universe whereas dark energy contributes two-third of the total energy density of the universe.\\
In fact, it was in 1920s when Alexander Friedmann and Georges Lemaître independently provided first cosmological model which explains that the universe is expanding. There are many different cosmological models which explain the expanding universe. One such model which is most favourable is $\Lambda$CDM model (Lambda cold dark matter) where $\Lambda$ is the cosmological constant. This model is referred to as the standard model of Big Bang cosmology but it suffers from fine tuning problem. Hence, there are many cosmological models which remains active areas of research in cosmology today and they all involve trying to understand the nature of dark matter and dark energy. 
%------------------------------------------------
\section{Scope of this report in a nutshell}

The absolute scope of this project is to find the performance of a novel cosmological model with an new branch of astrophysics. The project will pave a way to test the credibility of General Relativity with the Cosmology as the terms like Redshift and Luminosity Distance of the Gravitational Wave Merger Events are determined from our knowledge on General Relativity. Meanwhile the performance of Variable Chaplygin Gas in both Supernovae Dataset and Gravitational Waves Merger Event Dataset and comparing the results obtained from them helps us to calibrate our current understanding on the universe.   