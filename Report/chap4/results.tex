
\chapter{Statistical Analysis}
\label{sec:3}
\section{Overview}
In order to find the right combination of the constrains $\Omega_{m}$ and $n$ for which the model provides best fit of the luminosity distance and the distance modulus, a $\chi^{2}$ test is conducted between the observed values in the dataset and the calculated values from the model. $\chi$-squared is a test to verify a hypothesis proposed to explain the distribution in a given dataset. The hypothesis or the equation that is devised to fit the distribution of the data-points from given parameters are generally termed as Null Hypothesis. These null hypothesis when tested with a $\chi^{2}$ test provides a numerical value which defines the goodness of the hypothesis. This numerical value should be equal to or be around the numerical value of the total number of data-points involved in the dataset. $\chi$-squared test provides a clear picture on the performance of the model.   


\section{$\chi^2$ Test}
The standard equation to determine the goodness value of the $\chi^{2}$ test for distance modulus is given by
\begin{equation}
    \chi^{2} = \sum_{i}\left[\frac{\mu^{i}_{th}-\mu^{i}_{obs}}{\sigma_{i}}\right]
\end{equation}
where $\mu_{th}$ is the distance modulus value obtained from the theoretical calculation of the cosmological model and $\mu_{obs}$ is the observed value of distance modulus in the dataset. $\sigma$ is the possible magnitude of error in the observed distance modulus given by the dataset. The $\left[\frac{\mu_{th}-\mu_{obs}}{\sigma}\right]$ is calculated for each data-point and is summed for the total dataset.

In order to make to attribute the performance of other non constrained factors like the Hubble Parameter $H_0$, the equation is modified as
\begin{equation}
    \chi^{2} = \sum_{i}\left[\frac{\mu^{i}_{th}-\mu^{i}_{obs}}{\sigma_{i}}\right]-\frac{C_1}{C_2}{\left(C_1+\frac{2}{5}\ln{10}\right)}-2{\ln{h}}
\end{equation}
The terms $C_1$ and $C_2$ are given by
\begin{equation}
    C_1 = \sum_{i}\frac{\mu^{i}_{th}-\mu^{i}_{obs}}{\sigma_i^{2}}
\end{equation}
\begin{equation}
    C_2 = \sum_{i}\frac{1}{\sigma_i^{2}}
\end{equation}
$h$ is the dimensionless Hubble Parameter = $\frac{H_0}{100 (km)(s^{-1})(Mpc^{-1})}$


\section{Bibliography Notes}
Section 3.2 is influenced from \citep{bertibangalore} and rest of the sections were influenced by \citep{RevModPhys.70.1545} and \citep{nollert1999quasinormal}.