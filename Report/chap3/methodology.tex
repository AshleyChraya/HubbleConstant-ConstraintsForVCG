\chapter{Standard Candles and Standard Sirens}
\label{sec:4} 
\section{Overview}
In order to peek further into space, we need to make use of the global properties of galaxies, for small-scale distance measurement methods are rendered useless here. This is where standard candles come into the picture. Standard candles are fictitious objects of constant luminosity for which apparent magnitude is directly related to distance. Supernova is a high-energy phenomena that takes place at the end of stellar evolution. Even-though the discrete objects in the far away galaxies become faint and hard to resolve, a supernova explosion makes an exception.The consistent luminosity curve and relatively homogeneous properties of a type Ia supernova make it the perfect choice of standard candle for a cosmologist. 

Standard sirens are excellent distance measurement tools. The standardized amplitude gives us the idea of how far the source is. The gravitational waves from the Merger Events are ripples in the spacetime fabric which due to the nature of the medium it is traversed, doesn't lose it's energy with interaction with any gravitating object, be it baryonic matter or exotic matter like dark matter - dark energy. 

Testing the performance of Variable Chaplygin Gas Model in two different energy spectrum, the electromagnetic radiations (Standard Candles) and gravitational waves (Standard Sirens) which are originated from two different events is a solid way to test the ability of the model to mimic the nature of the Universe. 
   
\section{Type Ia Supernova data set and calibration}
The SCP (Supernova Cosmology Project) Union 2.1 dataset is compilation of 833 Type Ia Supernovae events collected and merged from 19 individual dataset. We have taken 580 events from the overall dataset, as these events are completely verified and could be trusted with higher level of confidence. The dataset is consist of data like redshift obtained from the event, distance modulus of the event and the potential error factor involved in estimation of the distance modulus.

The luminosity distance of the Supernovae event could be expressed as a function of redshift of the event pertaining to the Variable Chaplygin Gas Model. In a flat universe in which the parameter are constrained by the Variable Chaplygin Gas Model, the luminosity distance can be express as  
\begin{equation}\label{3 Luminosity distance equation}
    d_L(z,\textbf{p}) = c{(1+z)}{\int_{0}^{z}\frac{dz'}{H(z',\textbf{p})}}
\end{equation}
where z is the redshift, H is given by 
\begin{equation}\label{3 H term}
    H^{2} = \frac{8{\pi}G}{3}({\rho_{r0}{(1+z)}^{4}+\rho_{b0}{(1+z)}^{3}+\rho_{ch0}{[{\Omega_{m}{(1+z)}^{6}+{(1-\Omega_{m}}){(1+z)}^{n}}]}^{1/2}})
\end{equation}
where $\rho_{r0}$ and $\rho_{b0}$ are the current energy densities of radiation and baryons in the universe. $\rho_{ch0}$ is the energy density of Variable Chaplygin Gas entity consist of Dark Matter and Dark Energy. The \textbf{p} in the previous equation denotes all other parameters of the given cosmological model.   

\section{Overview about the GW Merger Events Dataset}
The Gravitational Merger events are obtained from the GWOSC (Gravitational Wave Open Science Center) which has the events obtained from detectors at LIGO Hanford, LIGO Livingston and LIGO Virgo. The data set consist of many confirmed events and potentially true events which are yet to be confirmed in it. The events are collected across all the three runs: O1 (from 12 September 2015 to 19 January 2016), O2 (from 30 November 2016 to 25 August 2017) and the O3 runs, O3a (from 1 April 2019 to 30 September 2019) and O3b (from 1 November 2019 to March 2020).

There were in total 53 confirmed events in the currently data set. All these confirmed events were taken to test the efficiency of the Variable Chaplygin Gas Model to predict the luminosity distance of the event from the redshift obtained from the gravitational merger events and compare with luminosity distance obtained from the merger event by analysis of the wave received at the detector.

\section{Luminosity Distance from Merger Events}
When a gravitational wave passes through the laser interferometer, it elongates the distance between the source and the reflector in the end in one arm and contracts the reflector length from the source in the other arm which is almost perpendicular to the elongated arm. This alternative elongation and contraction in the perpendicular arms, due to the quadrupole nature of the gravitational waves, causes path difference between the laser beams running between the arms and results in interference. The beams from perpendicular arms are made to interfere destructively in the standalone state but produce light by the interference induced by the passage of gravitational waves. The extent of elongation or contraction is determined by the amplitude of the gravitational wave passing the detector.

Since gravitational waves distort the geometrical length between the arms to cause interference, the amplitude of the laser beam from the interference pattern is directly related to the amplitude of the gravitational wave signal from the merger event. The gravitational waves weakly interact with matter, so the amplitude measured in the detector is an absolute quantity that one could measure from the merger events which provides insights about the events.

The amplitude of the wave/signal from the merger event is a function of linear separation between the binary, angular velocity of the system and also inversely proportional to the distance at which the event has occurred. This amplitude is a scale to quantify the energy emitted from the merger event as gravitational waves. Therefore we can express Luminosity of the merger event as a function of angular velocity, angular separation and distance. This luminosity is analogous to the luminosity described for stellar objects in electromagnetic counterparts.
\begin{equation}\label{AmplitudeGW}
    h = {\frac{G^{5/3}}{c^{4}}}\frac{{\mu}{a^{2}}{\omega^{2}}}{r}
\end{equation}
where, h is the Amplitude of the wave , $\mu$ is the reduced mass of the system = $\frac{{m_1}{m_2}}{m_1+m_2}$, $m_1$ and $m_2$ are the mass which are part of the binary system, a is separation between masses, r-distance between observer and $\omega$ - orbital velocity.

\begin{equation}\label{luminosity}
L = \frac{dE_{GW}}{dt} \approx \frac{G}{c^{5}}{h^{2}}{\omega^{2}} \approx \frac{G}{c^{5}}{\mu^{2}}{a^{4}}{\omega^{6}}
\end{equation} 
where, $\frac{dE_{GW}}{dt}$ is the rate of energy emitted by the binary event as gravitational waves.

Luminosity is a measure of energy emission from a given source. In case of binary merger events, the energy is transferred from the lost orbital energy of the binary system as the masses inspiral towards each other. So, the ratio between the frequency and the rate of change of frequency is proportional to the ratio between the separation of the mass in the system and the rate of change of this separation at successive orbits.

\begin{equation}\label{ratio frequency and amplitude}
	\frac{\omega}{\dot{\omega}} = \frac{-3}{2}{\frac{a}{\dot{a}}} = \frac{f}{\dot{f}}
\end{equation}
where, f = $\frac{\omega}{\pi}$ is the frequency of the signal and $\dot{f}$ = $\frac{\dot{\omega}}{\pi}$ is the rate of change in frequency

The ratio of frequency to the rate of change of frequency can be found from the waveform of the signal. The iconic chirp waveform of the gravitational wave from the detector helps us in obtaining the maximum amplitude, frequency and its ratio to the rate of change of frequency. From the above information, the luminosity distance (in Megaparsec (Mpc)) of a merger event can be found:
\begin{equation}\label{Luminosity distance}
		R = \frac{512}{h_{21}}{(\frac{0.01}{\tau})}{(\frac{100 Hz}{f_{GW}})}^{2}
\end{equation}

the $\tau$ is the subtle expression for the ratio of the frequency of the gravitational wave to the rate of change in the frequency as the separation between the masses of the binary system gets reduced as they inspiral towards each other  
\begin{equation}\label{tau}
        \tau = \frac{f_{GW}}{\dot{f_{GW}}}
\end{equation}
where, R - luminosity distance, $h_{21}$ - amplitude caused by the inspiral of mass 1 and mass 2 of the binary system.  
\section{Distance Modulus - Flat $\Lambda$CDM Model and VCG Model}
The distance modulus is a logarithmic scaling term used to describe the distance of an energy radiating entity in astronomy. The distance modulus $\mu$ can be expressed as the difference between the apparent magnitude ($m$) and the absolute magnitude ($M$). 

\subsection{Supernova Type Ia dataset}

The distance modulus that is provided in the SNe-Ia dataset is given by the $\Lambda$CDM Model for a given redshift as
\begin{equation}
    \mu_{obs} = {a}(t_{0})r{\left(1+z\right)}
\end{equation}
where a is the scale factor. The scale factor (a) in a flat universe can be expressed as $\frac{1}{(1+z)}$. $t_{0}$ is the Hubble Time which is $\frac{1}{H_0}$, where $H_0$ is the Hubble Parameter value in the current epoch. z is the redshift value of the entity under study. 

The r in Eq. (3.8) is the coordinate distance. In a flat $\Lambda$CDM model universe, the r is expressed as
\begin{equation}
    r = \int_{t}^{t_0}\frac{cdt}{a(t)} = \frac{c}{a_0H_0}\int_{0}^{z}\frac{dz'}{h(z')}
\end{equation}
The $a_0$ is the value of the scale factor in the observer's immediate surrounding. It is found to numerically equal to 1. The $h$ is Hubble parameter value for the region of space of the given redshift. The $h$ is expressed as 
\begin{equation}
    h(z) = [\left(1-\Omega_{total})(1+z)^{2}+\Omega_{m}(1+z)^{3}+\Omega_{\Lambda}(1+z)^{p}\right]^{1/2}
\end{equation}
in a flat universe, $\Omega_{total}$ = 1 and p is numerically equal to 1. So, the equation becomes
\begin{equation}
    h(z)={[\left\Omega_{m}(1+z)^{3}+\Omega_{\Lambda}\right]}^{1/2}
\end{equation}
so,
\begin{equation}
\mu_{obs} = \frac{1}{H_{0}^{2}(1+z)}\frac{c}{a_0}\int_{0}^{z}\frac{dz'}{[\left\Omega_{m}(1+z')^{3}+\Omega_{\Lambda}\right]^{1/2}}
\end{equation}
the reason why the redshift (z) inside the integral is dashed is to differentiate it with the redshift term outside the integration, and to denote that the denominator of the integral as a variable entity unlike the redshift term outside the integral.

The redshift of the given supernova event is used to calculate the distance modulus of the event and it is available in the SCP dataset.

With the Variable Chaplygin Gas Model, the distance modulus is given as
\begin{equation}
    \mu_{th} = 5\log{d_{L}(z)}-5\log{h}+42.38 = 5\log{d_{L}(z)}+5log\left(\frac{cH_{0}}{1Mpc}\right)+25
 \end{equation}
the $d_{L}$ is obtained from equation 3.1 

\subsection{Gravitational Wave Merger dataset}
The standard distance modulus of the merger event is calculated from the luminosity distance obtained by the signal of the merger event as described in section 3.4 as
\begin{equation}
    \mu_{obs} = 5\log{d_{L}}-5
\end{equation}
where $d_{L}$ is the distance modulus obtained from the wave analysis. 

The $\mu_{th}$ part is obtained from Equation 3.13, where the $d_{L}$ is a function of redshift given by the wave analysis of signals from merger events. The expression for $d_{L}$ is given at Equation 3.1.

The $\mu_{obs}$ is termed as observed distance modulus that is given in the database from standard cosmological model, the $\Lambda$CDM Model. The $\mu_{th}$ is the theoretical distance model, obtained from Variable Chaplygin Gas (VCG) Model.

\section{Bibliography Notes}
Section 3.4 was influenced from \citep{bertibangalore}, section 4.3 was influenced by  \citep{leaver1985analytic}, and section 4.4 was heavily influenced by \citep{gurbir}.
