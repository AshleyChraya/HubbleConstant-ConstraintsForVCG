\chapter{Variable Chaplygin Gas}
\label{sec:2}
\section{Overview}
In 1920s, Alexander Friedmann theorized that the Universe is expanding using Einstein field equations after Vesto Slipher identified redshift of many galaxies. This theory was proved by the observational evidences from Edwin Hubble. George Lemaitre provided the linear relation between the distance of the galaxy and the redshift observed in them due to expansion of the universe. 
In early 1990s, with the observations of type Ia Supernovae (SNeIa) by independent research groups Supernova Cosmology Project (SCP) and the High-Z Supernova Search, the expansion of the universe was found to be accelerated compared to the initial assumption of the linear expansion by George Lemaitre. By the Friedmann-Robertson-Walker metric of the Universe, the expansion is directly proportional to the amount of matter present in the Universe. With accelerated expansion, the universe needed additional presence of mass or energy to be explained with the current cosmological model. With no evidence on these additional matter from Electromagnetic Radiations, this additional energy termed as Dark Energy. 

Similarly in 1930s, Fritz Zwicky found it in the Coma cluster that mass of the galaxies has to be higher than the visible matter to validate the higher orbital speed of the periphery of the galaxies. He proposed a presence of electromagnetically "dark matter" causing such anomalous high speed of rotations. This was proved by observations of Vera Rubin and others in 1970s and the theory of Dark Matter came into consideration.  

The pressing need to explain such phenomenon resulted in theorizing models which clubbed in these exotic entities in them. The presence of these entities brought in the $\Lambda$CDM ($\Lambda$ Cold Dark Matter) Model where $\Lambda$ denotes the Dark Matter and Dark Energy being the dominant entities in the universe under this model. With new insights on the universe, like Inflation Epoch and also with problems like fine tuning problem, the model has to modified to a more conclusive models validating the observations. One such model is the Chaplygin Gas Model.
 
\section{Modified gravity and Cosmological models}
All different models can be categorized into two classes. One class of models try to modify the geometry part of the Einstein's equation i.e. Einstein tensor $G_{ij}$. These theories are known as modified gravity. These are generalizations of the general relativity (GR) where gravity action is modified from $ A_{g}=\frac{-1}{16 \pi k} \int R \sqrt{-g} d^{4} x$ to $ A_{mg}=\frac{-1}{16 \pi k} \int f(R) \sqrt{-g} d^{4} x$ where $f(R)$ is a function of Ricci scalar,$A_g$ is the action for GR, $A_{mg}$ is the action for modified gravity. One can also work in higher dimensions instead of four as a part of modifying gravity. The other class of models assumes universe is governed by general relativity but alter the matter component of the Einstein's equation i.e. Energy-Momentum tensor $T_{ij}$. Models in both classes explain the expanding universe. It is not possible to explain the accelerated universe if only normal matter is taken into consideration. Hence, exotic matter i.e. dark matter has to be added to the energy momentum tensor. Models based on this dark matter can be further divided into quintessence models where slowly evolving and spatially homogeneous scalar field or two coupled fields are considered and dark energy models including barotropic fluids. Quintessence models also suffer from fine tuning problem, therefore, possible alternative which we should look into is dark energy models like barotropic fluids.\\
%--------------------------------------------------------------------------------------------------------------------------------------
\subsection{Dark Energy vs Normal Matter}\label{sec2}
The dark energy in contrast to normal matter has the negative pressure.



\section{Chaplygin Gas Model}
Dark energy models including barotropic fluids have pressure as a function of energy density, $P=f(\rho)$, which determines the dynamics of the fluid. These models also contains classes with varying equation of state. One specific example of the barotropic fluid is Chaplygin gas (CG) whose equation of state is $P=-A/\rho$, where A is a positive constant. Chaplygin gas  model  correctly describes the effects  of  dark  energy  and  dark  matter,  and  is a alternate for our current standard model of cosmology. As described in the previous section dark energy models have negative pressure, CG model also has a negative pressure associated with it and the more generalized model of CG is described by the equation of the sate given as
\begin{equation}\label{ch}
    p=-\frac{A}{\rho ^\alpha}
\end{equation}
where p is the pressure, $\rho$ is the energy density, both in a comoving reference frame with $\rho>0$, A and $\alpha$ ($0<\alpha \leq 1$) are positive constants ($\alpha=1$ corresponds to CG). In FLRW metric, energy conservation equation is given by
\begin{equation}\label{conseq}
    \dot{\rho}_{i}+\frac{3 \dot{a}}{a}\left(p_{i}+\rho_{i}\right)=0
\end{equation}
This is the fluid equation which holds for radiation(r), Baryonic matter(b) and chaplygin gas(ch), i.e. $i={r,b,ch}$. In this report, we assume the universe is filled with CG-radiation-baryonic matter. Using \ref{ch} and \ref{conseq}, we can get the expression for the energy density of the CG given by 
\begin{equation}\label{CGdensity}
    \rho_{ch}=\left(A+\frac{B}{a^{3(1+\alpha)}}\right)^{\frac{1}{1+\alpha}}
\end{equation}
where B is the constant of integration and a is the scale factor. Hence, we got the equation of state and energy density of the CG model given by \ref{ch} and \ref{CGdensity} respectively. We can similarly find the  energy density for baryonic matter and radiation given their equation of state. \\
Equation of state for baryonic matter is $p=0$. Using energy conservation equation \ref{conseq} we can obtain the energy density of the baryonic matter given by 
\begin{equation}\label{bar}
    \rho_b=\rho_{b0} a^{-3}
\end{equation}
where $\rho_{b0}$ is the integration constant. Equation of state for the radiation is $p=\rho/3$. Radiation is the ideal fluid, hence, energy momentum tensor in curved spacetime can be written as $T^i_j=(p+\rho) U^i U_j - p g^i_j$. Taking trace of this equation we get $T^i_j= p+\rho -4p=\rho-3p$. Using equation of state we get the trace of the energy momentum tensor for the radiation to vanish. This vanishing of trace of $T_I_j$ is a common feature for theories which are conformally invariant. Anyways, repeating the same story which we did for CG and baryonic matter i.e using energy conservation equation \ref{conseq}, we can obtain the energy density of the radiation which is given by
\begin{equation}\label{rad}
    \rho_r=\rho_{r0}a^{-4}
\end{equation}
Total energy density can be written as the sum of the components \ref{CGdensity}, \ref{bar}, \ref{rad}, given by
\begin{equation}\label{tot}
    \rho_{total}(a)=\left(A+\frac{B}{a^{3(1+\alpha)}}\right)^{\frac{1}{1+\alpha}} + \rho_{b0} a^{-3} + \rho_{r0}a^{-4}
\end{equation}
The Friedmann equation gives the expansion rate of the Universe in terms of matter and radiation density,$\rho$, curvature,k, and the cosmological constant,$\Lambda$, as
\begin{equation}\label{freidmann}
    H^{2} \equiv\left(\frac{\dot{a}}{a}\right)^{2}=\frac{8 \pi G}{3} \rho-\frac{k}{a^{2}}+\frac{\Lambda}{3}
\end{equation}
where $H\equiv \frac{\dot a}{a}$ is the Hubble parameter. Assuming spatially flat universe, we get
\begin{equation}
    H^{2}=\frac{8 \pi G}{3} \rho
\end{equation}
Using total energy density expression \ref{tot}, Freidmann equation \ref{freidmann} reads as
\begin{equation}
    3 H^2= 8\pi G\left(A+\frac{B}{a^{3(1+\alpha)}}\right)^{\frac{1}{1+\alpha}} + \rho_{b0} a^{-3} + \rho_{r0}a^{-4}
\end{equation}
We can also make change of variables and use redshift, z, instead of using scale factor, a, which is not a physically measurable quantity. The relation between redshift and scale factor is given by
\begin{equation}\label{RDZ}
    1+z=\frac{a_0}{a}
\end{equation}
where $a_0$ is the scale factor at the present time which we normalize to 1. Freidmann's equation in terms of redshift can be written as
\begin{equation}\label{Freidredshift}
3 H^{2}(z)=\left\{\left[A+B(1+z)^{3(1+\alpha)}\right]^{\frac{1}{1+a}}+\left(\rho_{r0}\right)(1+z)^{4}+\left(\rho_{b0}\right)(1+z)^{3}\right\}
\end{equation}
\subsection{Evolution of Chaplygin Gas}
We can determine the evolution of the energy density of the CG during different epochs by analyzing equation \ref{CGdensity}. At early times, $a<<1$, \ref{CGdensity} can be written as 
\begin{equation}
    \rho_{ch}=\frac{B^{1/(1+\alpha)}}{a^3}
\end{equation}
Original CG energy density by using $\alpha=1$ can be written as
\begin{equation}
      \rho_{ch}=\frac{\sqrt{B}}{a^3}
\end{equation}
Therefore, at early times OCG corresponds to dust like matter(or dark matter). On the other hand, at late times we can similarly show that
\begin{equation}
    \rho_{ch}= -p = A^{1/(1+\alpha)} = Constant
\end{equation}
Following the discussion in \ref{sec2}, we can conclude that CG at late times corresponds to a cosmological constant. Thus leads to the observed accelerated expansion. Therefore, we can further conclude that CG evolves from the dust dominated epoch to cosmological constant in present times, and thus CG model can unify CDM and the $\Lambda$ model features. Therefore, the CG model is a good alternative to explain the accelerated expansion of the universe. However the CG model produces an exponential blowup of matter power spectrums that are inconsistent with observations. Due to this, a modification of the CG model is proposed called Variable Chaplygin Gas model.
\section{Variable Chaplygin Gas Model}
Equation of state of the Variable Chaplygin Gas (VCG) is given by
\begin{equation}\label{VCG}
    \rho_{ch}=\frac{A(a)}{p}
\end{equation}
where where p is the pressure, $\rho$ is the energy density, both in a comoving reference frame with $\rho>0$, $A(a) = A_0 a^{-n}$ is a positive function of the cosmological scale factor a. $A_0$ and n are constants. Using \ref{conseq}, we can find the energy density of VCG as
\begin{equation}\label{EDVCG}
    \rho_{c h}=\sqrt{\frac{6}{6-n} \frac{A_{0}}{a^{n}}+\frac{B}{a^{6}}}
\end{equation}
where B is the positive integration constant. For n=0, OGC behaviour is recovered. Assuming universe to be spatially flat, 2nd Freindmann equation can be written as 
\begin{equation}
    2\frac{\ddot a}{a} + H^2=-\frac{8 \pi G}{c^2} p
\end{equation}
The acceleration condition, $\ddot a$ can be written as
\begin{equation}
    (H^2 +\frac{8\pi G}{c^2}p)a<0
\end{equation} 
Using equation of state of VCG \ref{VCG} and energy density of VCG \ref{EDVCG}, we find the above acceleration condition is equivalent to 
\begin{equation}
    3 \frac{4-n}{6-n} a^{6-n} >\frac{B}{A_0}
\end{equation}
As both B and $A_0 $ are positive constants, hence $n<4$. At present time, a=$a_0 $= 1, hence, the present value of the energy density of VCG is given by
\begin{equation}
    \rho_{ch0}=\sqrt{\frac{6}{6-n} A_{0}+B}
\end{equation}
Defining the parameter, $\Omega_m$,
\begin{equation}
    \Omega_{m}=\frac{B}{6 A_{0} /(6-n)+B}
\end{equation}
the energy density becomes
\begin{equation}\label{VCGdensity}
    \rho_{ch}(a)=\rho_{c h 0}\left[\frac{\Omega_{m}}{a^{6}}+\frac{1-\Omega_{m}}{a^{n}}\right]^{1 / 2}
\end{equation}

Total energy density of the universe can be written as the sum of components \ref{VCGdensity}, \ref{bar}, \ref{rad}, given by
\begin{equation}\label{totVCG}
    \rho_{total}(a)=\left(\rho_{c h 0}\left[\frac{\Omega_{m}}{a^{6}}+\frac{1-\Omega_{m}}{a^{n}}\right]^{1 / 2}\right) + \rho_{b0} a^{-3} + \rho_{r0}a^{-4}
\end{equation}
Defining $\Omega_{r0} = \frac{8 \pi G\rho_{r0}}{3H_0 ^2}$ and $\Omega_{b0} = \frac{8 \pi G\rho_{b0}}{3H_0 ^2}$, $\Omega_{ch0} = \frac{8 \pi G\rho_{ch0}}{3H_0 ^2}$ as dimensionless density parameters. The density parameters for radiation and baryonic matter can be expressed as
\begin{equation}
    \Omega_{r}(z)=\left(\Omega_{r0}\right)(1+z)^{4}, \quad \Omega_{b}(z)=\left(\Omega_{b0}\right)(1+z)^{3}
\end{equation}The total density parameter for a universe where CG, baryonic matter and radiation dominate can be written as
\begin{equation}
    1=\Omega_{r}(z)+\Omega_{b}(z)+\Omega_{e h}(z)
\end{equation}
Using total energy density expression \ref{totVCG} and \ref{RDZ}, Freidmann equation \ref{freidmann} in terms of redshift reads as
\begin{equation}
    H^2=\Omega_{ch0} H_0^2 (1+z)^4 X^2(z)
\end{equation}
where
\begin{equation}\label{Imp}
    X^{2}(z)=\frac{\Omega_{r 0}}{1-\Omega_{r 0}-\Omega_{b 0}}+\frac{\Omega_{b 0} }{1-\Omega_{r 0}-\Omega_{\theta 0}(1+z)}+\frac{\left(\Omega_{m}(1+z)^6+(1-\Omega_{m})(1+z)^n\right)^{1 / 2}}{(1+z)^4}
\end{equation}
To test VCG model, the above equation \ref{Imp} is useful. There are two free paramters in the above equation $\Omega_m$ and n. We can use the distance modulus,$\mu$, for Supernovae Type IA data and gravitational waves from compact binary coalescence's (CBCs), and calculate the corresponding distance modulus for the CG model at corresponding redshifts.








%-----------------------------------------------------------------------------------------------------------------------------------------







\section{Bibliography Notes}
This whole chapter was greatly influenced by chapter 2 from \citep{bertibangalore} and \citep{carroll2019spacetime}.





